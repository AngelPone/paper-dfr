\documentclass[12pt]{beamer}

\usepackage{xcolor}
\usepackage{layouts}
\usepackage{minted}
\usepackage[most]{tcolorbox}
\usepackage{natbib}
\tcbuselibrary{minted}
\usemintedstyle{xcode}
\setminted{
    fontfamily=helvetica
}

\usepackage{booktabs}

\usepackage{outlines}
\usepackage{hyperref}
\usepackage{graphicx}
\usepackage{amsmath}
\usepackage{makecell}

% justify
\usepackage{ragged2e}
\justifying\let\raggedright\justifying


% outline space
\makeatletter
\renewenvironment{outline}[1][]{%
  \ifthenelse{\equal{#1}{}}{}{\renewcommand{\ol@type}{#1}}%
  \ol@z%
  \newcommand{\0}{\ol@toz\ol@z}%
  \newcommand{\1}{\vspace{\dimexpr\outlinespacingscalar\baselineskip-\baselineskip}\ol@toi\ol@i\item}%
  \newcommand{\2}{\ol@toii\ol@ii\item}%
  \newcommand{\3}{\ol@toiii\ol@iii\item}%
  \newcommand{\4}{\ol@toiiii\ol@iiii\item}%
}{%
  \ol@toz\ol@exit%
}
\makeatother




\renewcommand\theadalign{bc}
\renewcommand\theadfont{\bfseries}
\renewcommand\theadgape{\Gape[4pt]}
\renewcommand\cellgape{\Gape[4pt]}

\beamertemplatenavigationsymbolsempty

\newif\ifsectiontitlepage
\sectiontitlepagetrue

\newcommand{\nosectiontitlepage}{\sectiontitlepagefalse}


\AtBeginSection[]{
\ifsectiontitlepage
  \begin{frame}
  \tableofcontents[currentsection, hideothersubsections]
  \end{frame}
\fi
}



%%

% color
\definecolor{coreFontMain}{RGB}{26,26,26}
\definecolor{coreFontSub}{RGB}{106,106,106}
\definecolor{coreBackgroundLight}{RGB}{230,230,230}
\definecolor{coreBackground}{RGB}{255,255,255}

% heading
\newlength{\headerheight}
\setlength{\headerheight}{0.094\paperheight}


\setbeamerfont{frametitle}{size=\fontsize{12}{14.4}}
\setbeamertemplate{frametitle}
{%
\nointerlineskip
\begin{beamercolorbox}[wd=\paperwidth,left,sep=0pt]{}
\begin{tcolorbox}[enhanced,frame hidden, 
    borderline south={0.5pt}{0pt}{coreFontSub},
    colback=coreBackground, coltext=coreFontSub,height=\headerheight,notitle,left*=0.05\paperwidth,boxsep=0pt,bottom=0pt,halign=left,valign=bottom,before skip=0pt,after skip=0pt, before upper=\strut]
    \usebeamercolor{frametitle}\insertframetitle
\end{tcolorbox}
\end{beamercolorbox}
}


% footer
\newlength{\footerheight}
\setlength{\footerheight}{0.06\paperheight}
\setbeamercolor{footer}{bg=coreBackgroundLight, fg=coreFontMain}

\setbeamerfont{footline}{size=\fontsize{10}{12}}
\setbeamertemplate{footline}
{%
\begin{beamercolorbox}[sep=0pt,wd=\paperwidth,center]{footer}
%\begin{minipage}[c][\footerheight]{\textwidth}
\begin{columns}
    \begin{column}{0.7\textwidth}
        \begin{minipage}[c][\footerheight]{\textwidth}
            {\insertsectionhead}             
        \end{minipage}
    \end{column}
    \begin{column}{0.1\textwidth}
        \begin{minipage}[c][\footerheight]{\textwidth}
            \centering
            \insertframenumber/\inserttotalframenumber            
        \end{minipage}
    \end{column}
\end{columns}
    
%\end{minipage}
\end{beamercolorbox}
}



% Title Page

% header
% \newlength{\titleheaderheight}
% \setlength{\titleheaderheight}{0.0666667\paperheight}
% \setbeamercolor{header}{bg=coreBackgroundLight}

%% logo
\newlength{\logoheight}
\setlength{\logoheight}{0.297\paperheight}

%% title
\setbeamercolor{title}{fg=coreFontMain}
\setbeamerfont{title}{size=\fontsize{18.14}{21.768}}
\newlength{\titleheight}
\setlength{\titleheight}{0.19\paperheight}


%% meta
\newlength{\metaheight}
\setlength{\metaheight}{0.21\paperheight}
\setbeamercolor{meta}{fg=coreFontSub}
\setbeamerfont{meta}{size=\fontsize{9.07}{10.884}}

%% contact
\newlength{\contactheight}
\setlength{\contactheight}{0.193333\paperheight}



\makeatletter
% \newcommand{\iftitlegraphicisempty}{%
%   \ifx\@empty\inserttitlegraphic%
%     \expandafter\@firstoftwo%
%   \else%
%     \expandafter\@secondoftwo%
%   \fi%
% }
\setbeamertemplate{title page}{
% \nointerlineskip
% 
\ifx\@empty\inserttitlegraphic%
    \setlength{\titleheight}{0.4\paperheight}
    \setlength{\metaheight}{0.4\paperheight}%
    \setbeamerfont{meta}{size=\fontsize{11}{13.2}}
\else%
    \inserttitlegraphic%
\fi%

\begin{tcolorbox}[enhanced,frame hidden, borderline south={0.5pt}{0pt}{coreFontMain}, colback=coreBackground,height=\titleheight,boxsep=0pt,notitle,halign=flush center,valign=center, bottom=-30pt,before skip=0pt,after skip=0pt, coltext=coreFontMain]
\usebeamercolor{title}
\usebeamerfont{title}\inserttitle
\end{tcolorbox}

\usebeamercolor[fg]{meta}{
\begin{minipage}[c][\metaheight]{\textwidth}
    \usebeamerfont{meta}
    \centering
    \insertauthor \linebreak
    \insertinstitute \linebreak
    \insertdate
\end{minipage}
}


\begin{tcolorbox}[enhanced,frame hidden, colback=coreBackground,height=\contactheight,notitle,halign=flush center,valign=top,before skip=0pt,after skip=0pt,top=-5pt]
\begin{tabular}{ccc}
\href{mailto:zhangbohan@buaa.edu.cn}{\includegraphics[height=3mm]{bohanbeamerstyle/figures/mail.png}} & \href{https://github.com/angelpone}{\includegraphics[height=3mm]{bohanbeamerstyle/figures/github.png}} & \href{https://twitter.com/zhangbhangel}{\includegraphics[height=3mm]{bohanbeamerstyle/figures/Twitter.png}}
\end{tabular}

\end{tcolorbox}


% \begin{beamercolorbox}[sep=2pt,center,wd=\paperwidth,ht=\footerheight]{footer}
% \tiny
% \insertframenumber/\inserttotalframenumber
% \end{beamercolorbox}
% \nointerlineskip
}


% 
\setbeamercolor{normal text}{bg=coreBackground,fg=coreFontMain}

% itemize
\setbeamertemplate{itemize items}[circle]

\setlength{\leftmarginii}{10pt}
\setbeamercolor{itemize item}{fg=coreFontMain}
\setbeamercolor{itemize subitem}{fg=coreFontMain}
\setbeamercolor{itemize subsubitem}{fg=coreFontMain}



\usepackage{blkarray}
% \titlegraphic{ss}
\title{Discrete Forecast Reconciliation}
\author{Bohan Zhang}
\institute{Beihang University \\ 
IIF Workshop on Forecast Reconciliation}
\date{September 7, 2023}


% style setting
\nosectiontitlepage
\def\outlinespacingscalar{1.3}
\setbeamertemplate{blocks}[rounded][shadow=false]
\setbeamercolor{block title}{bg=teal,fg=white}
\setbeamercolor{block body}{bg=teal!10}
\setbeamerfont{block title}{size=\fontsize{10}{12}}

\begin{document}

\begin{frame}[plain]

    \maketitle

\end{frame}



\begin{frame}
    \frametitle{Collaborate with}
    \fontsize{9}{10.8}\selectfont
    \begin{table}
    \begin{tabular}{ccc}
        \includegraphics[width=0.3\textwidth]{figures/tas.jpg} &
        \includegraphics[width=0.3\textwidth]{figures/yfkang.png} &
        \includegraphics[width=0.3\textwidth]{figures/fengli.png} \\
        \begin{minipage}[t]{0.3\textwidth}\centering Anastasios Panagiotelis \\ University of Sydney \end{minipage} &
        \begin{minipage}[t]{0.3\textwidth}\centering Yanfei Kang \\ Beihang University \end{minipage} &
        \begin{minipage}[t]{0.3\textwidth}\centering Feng Li \\ Central University of \\ Finance and Economics \end{minipage}
    \end{tabular}
    \end{table}

\end{frame}


\begin{frame}
    \frametitle{CONTENTS}
    \tableofcontents
\end{frame}


\section{Introduction}
\begin{frame}
\frametitle{Motivation}
\framesubtitle{the problem}

\begin{outline}
\1 Non-negative and discrete-valued time series, particularly those with low counts, commonly arise in various fields. Examples include:
\2 occurrences of “black swan” events
\2 intermittent demand in the retail industry

\1 Despite the great concern of hierarchical forecasting in these applications, limited research have been conducted.


\end{outline}

\end{frame}

\begin{frame}
\frametitle{Motivation: the lesson learned from reconciliation approach}


\begin{outline}
    \0 The forecast reconciliation approach
    \1 first produces base forecasts for each series in the hierarchy; then optimally reconciles the base forecasts through projection;
    \1 utilises forecast combination, which improves forecast accuracy and reduces the risk of model misspecification;
    \1 has been shown to improve forecast accuracy in various applications.
    \0 But it was designed for continuous-valued HTS and can not be directly applied to discrete-valued HTS: \textit{projection} may produce non-integer and non-negative forecasts.
\end{outline}

\end{frame}


\begin{frame}
\frametitle{Motivation: probabilistic forecasting}
    \begin{outline}

        \1 While point and interval forecasts are most widely applied in practice, attention has been shifted towards full predictive distribution.
        \1 When forecasting discrete-valued time series, it is also more natural to produce predictive distribution.

    \end{outline}

\end{frame}


\begin{frame}{Related Work}

A series of work on forecast reconciliation for count HTSs:

\begin{itemize}
    \fontsize{9}{10.2}\selectfont
    \item Corani, G., Azzimonti, D., Rubattu, N., \& Antonucci, A. (2022). Probabilistic Reconciliation of Count Time Series (arXiv:2207.09322). arXiv.
    \item Zambon, L., Azzimonti, D., \& Corani, G. (2022). Efficient probabilistic reconciliation of forecasts for real-valued and count time series (arXiv:2210.02286). arXiv.
    \item Zambon, L., Agosto, A., Giudici, P., \& Corani, G. (2023). Properties of the reconciled distributions for Gaussian and count forecasts (arXiv:2303.15135). arXiv.
\end{itemize}

The proposed framework conditions base probabilistic forecasts of the most disaggregated series on base forecasts of aggregated series. However, it fails to restore the dependence structure within hierarchical time series.

\end{frame}

\begin{frame}{Contribution}

To address these concerns, we 

\begin{itemize}
    \item introduce the definition of \textit{coherent domain and coherent forecasts} in the context of multivariate discrete random variables.
    \item propose a discrete forecast reconciliation framework.
    \item develop the DFR and Stepwise DFR (SDFR) algorithms to train the reconciliation matrix.
    \item extend the top-down and bottom-up method to discrete probabilistic setting for comparison.
    \item verify the applicability of the algorithms in two simulation expriments and two real-world applications.
\end{itemize}

\end{frame}



\section{Coherent discrete predictive distribution}

\begin{frame}{Coherent and incoherent domain for discrete HTS}
\begin{table}
    \fontsize{10}{12}\selectfont
\begin{tabular}{ll}
    \toprule
    HTS & $\mathbf{Y}=(Y_1,Y_2,\dots,Y_n)'$ \\
    basis (e.g., bottom-level) time series & $\mathbf{Y}_b = (Y_1,Y_2,\dots,Y_{m})'$ \\
    domain of $i$-th variable & $\mathcal{D}(Y_i) = \{0,1,\dots,D_i\}$ \\\bottomrule
\end{tabular}
\end{table}


    Complete domain of $\mathbf{Y}$ is the Cartesian product of domains of all variables.
    \[
        \hat{\mathcal{D}}(\mathbf{Y}) = \{0,\dots,D_1\}\times \dots\times \{0,\dots,D_n\} \quad q:= |\hat{\mathcal{D}}(\mathbf{Y})|
    \]

    Coherent domain of $\mathbf{Y}$ is a subset of $\hat{\mathcal{D}}(\mathbf{Y})$, in which every point respects the aggregation constraints.
    \[
        \tilde{\mathcal{D}}(\mathbf{Y}) = \{\mathbf{y} | \mathbf{y}\in \hat{\mathcal{D}}(\mathbf{Y}), S\mathbf{y}_b = \mathbf{y}\} \quad r:=|\tilde{\mathcal{D}}(\mathbf{Y})|
    \]

    Incoherent domain of $\mathbf{Y}$
    \[
        \bar{\mathcal{D}}(\mathbf{Y}) = \hat{\mathcal{D}}(\mathbf{Y}) \backslash \tilde{\mathcal{D}}(\mathbf{Y})
    \]


\end{frame}


\begin{frame}{Example}
\begin{block}{Variables}
\[
\begin{aligned}
  \mathcal{D}(Y_1) = \{0, 1\}, \mathcal{D}(Y_2) = \{0, 1\}, \\
  Y_3=Y_1+Y_2, \mathcal{D}(Y_3)\in\{0,1,2\}
\end{aligned}
\]
\end{block}


\begin{block}{Complete domain}
\[
\begin{aligned}
\hat{\mathcal D}(\mathbf{Y})=&\left\{\mathbf{(0,0,0)'},(0,1,0)',(1,0,0)',(1,1,0)',\right.\\
&\left.(0,0,1)',\mathbf{(0,1,1)'},\mathbf{(1,0,1)'},(1,1,1)',\right.\\
&\left.(0,0,2)',(0,1,2)',(1,0,2)',\mathbf{(1,1,2)'}\right\}\,,
\end{aligned}
\]
\end{block}


\begin{block}{Coherent domain}
\[
    \tilde{\mathcal D}(\mathbf{Y})=\left\{(0,0,0)',(0,1,1)',(1,0,1)',(1,1,2)'\right\}\,.
\]    
    
\end{block}
\end{frame}

\begin{frame}{Discrete coherence}

\begin{definition}[Discrete Coherence]
A coherent discrete distribution has the property $Pr(\mathbf{Y} = 
\mathbf{y}) = 0, \forall y \in \bar{\mathcal{D}}(\mathbf{Y})$. Any distribution not meeting this condition is an incoherent distribution.
\end{definition}

\begin{outline}
    \1 We use a probability vector to represent the discrete predictive distribution.
    \1 Denote (potentially) incoherent base forecasts by $\hat{\boldsymbol{\pi}}$ and reconciled forecasts by $\tilde{\boldsymbol{\pi}}$.
    \[
        \begin{aligned}
      \hat{\boldsymbol{\pi}} &:= [\hat {\pi}_1, \hat\pi_2\dots, \hat \pi_q]' := [\hat{\pi}_{(y_1,\dots,y_n)^{(1)}}, \dots, \hat{\pi}_{(y_1,\dots,y_n)^{(q)}}] \\
      \tilde{\boldsymbol{\pi}} &:= [\tilde {\pi}_1, \tilde\pi_2\dots, \tilde \pi_r]' := [\tilde{\pi}_{(y_1,\dots,y_n)^{(1)}}, \dots, \tilde{\pi}_{(y_1,\dots,y_n)^{(r)}}]
        \end{aligned}
\]
\end{outline}

\end{frame}

\begin{frame}{Example}

\begin{block}{Base forecast}
    \[\begin{aligned}
            \hat{\boldsymbol{\pi}} = [\hat {\pi}_1, \hat\pi_2\dots, \hat \pi_{12}]' &= [\hat{\pi}_{(001)}, \hat{\pi}_{(011)}\dots, \hat{\pi}_{(112)}]' \\ &= [0.01, 0.02, \dots, 0.03]'       
        \end{aligned}\]
\end{block}

\begin{block}{Reconciled forecast}
\[\begin{aligned}
    \tilde{\boldsymbol{\pi}} = [\tilde {\pi}_1, \tilde\pi_2, \tilde\pi_3, \tilde\pi_4]'& = [\tilde{\pi}_{(000)}, \tilde{\pi}_{(011)}, \tilde{\pi}_{(101)}\tilde{\pi}_{(112)}]' \\ &= [0.2, 0.3, 0.4 0.1]'      
\end{aligned}  
\]
    
\end{block}

\end{frame}

\section{The discrete reconciliation framework}

\begin{frame}{The discrete reconciliation framework}

    \begin{outline}
    \0 \[\tilde{\boldsymbol\pi} = \psi(\hat{\boldsymbol{\pi}})\quad \psi:[0,1]^{q} \rightarrow [0,1]^r\]
    
    \0 The linear reconciliation function
    \[
        \tilde{\boldsymbol\pi} = \boldsymbol{A}\hat{\boldsymbol{\pi}}
    \] where, $\boldsymbol{A}=[a_{ij}], i=1,\dots,r,j=1,\dots,q$ is an $r \times q$ \textit{reconciliation matrix} with following constraints:
    \[\begin{aligned}
        0 \leq a_{ij} \leq 1, &\forall i, j\\
        \sum_{i=1}^r a_{ij} = 1, &\forall j
    \end{aligned}\]
    The framework reconciles the base forecasts by proportionally assigning the probability of each point in complete domain to points in the coherent domain.
    \end{outline}
\end{frame}

\begin{frame}{The discrete reconciliation framework}

\begin{block}{Example}
    \resizebox{\textwidth}{!}{
\begin{blockarray}{rcccccccccccc}
    & 000 & 010 & 100 & 110  & 001  & 011 & 101 & 111 & 002  & 012 & 102 & 112 \\
\begin{block}{r[cccccccccccc]}
000 & 0   & 0.4 & 0.3 & 0.25 & 0    & 0   & 0   & 0   & 0.3 & 0   & 1   & 0.2   \\
011 & 0   & 0.4 & 0.3 & 0.25 & 0.2  & 0   & 0   & 1   & 0.3 & 1   & 0   & 0.3   \\
101 & 0   & 0   & 0.3 & 0.25 & 0.4  & 0   & 1   & 0   & 0.3 & 0   & 0   & 0.5   \\
112 & 1   & 0.2 & 0.1 & 0.25 & 0.4  & 1   & 0   & 0   & 0.1 & 0   & 0   & 0     \\
\end{block}
\end{blockarray}}            
\end{block}
\begin{outline}
    \1 The framework allows the probability of a point in the complete domain assigned to \text{any} point in the coherent domain. 
    \1 For example, in an extreme case, from a coherent point $(000)$ to another coherent point $(112)$.
\end{outline}
    
\end{frame}

\begin{frame}{Movement restriction}

\begin{outline}
\0 Movement restriction strategy requires that the probability is only assgined to \textbf{the cloest coherent points}.
\1 This is similar to the projection idea in the optimal combination reconciliation framework.    
\1 A coherent point in the complete domain moves all of its probability to the same point in the coherent domain.
\1 We choose the L1 norm as the distance measure.
\[
    d((0, 0, 0), (0, 0, 1)) = |(0,0,0) - (0,0,1)|_1 = 1  
\]
\end{outline}
\vspace{-3mm}
\begin{block}{Example}\resizebox{\textwidth}{!}{\begin{blockarray}{rcccccccccccc}
    & 000 & 010 & 100 & 110  & 001  & 011 & 101 & 111 & 002  & 012 & 102 & 112 \\
\begin{block}{r[cccccccccccc]}
000 & 1   & 0.4 & 0.3 & 0.25 & 0.4  & 0   & 0   & 0   & 0.3 & 0    & 0   & 0   \\
011 & 0   & 0.6 & 0   & 0.25 & 0.3  & 1   & 0   & 0.3 & 0.3 & 0.35 & 0   & 0   \\
101 & 0   & 0   & 0.7 & 0.25 & 0.3  & 0   & 1   & 0.3 & 0.3 & 0    & 0.4 & 0   \\
112 & 0   & 0   & 0   & 0.25 & 0    & 0   & 0   & 0.4 & 0.1 & 0.65 & 0.6 & 1    \\\end{block}\end{blockarray}}\end{block}

\end{frame}


\section{Score-optimal algorithm: DFR}

\section{Addressing the dimensionality problem: SDFR}

\section{Simulation}

\section{Emprical study}

\section{Conclusion}

\begin{frame}{Advantage of the methods}
    
\end{frame}



\bibliography{../manuscript/references.bib}
\bibliographystyle{agsm}
\end{document}