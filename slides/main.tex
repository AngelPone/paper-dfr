\documentclass[12pt]{beamer}




%\directlua{ require("drawboxes")}
\usepackage{xcolor}
\usepackage{layouts}
\usepackage{minted}
\usepackage[most]{tcolorbox}
\usepackage{natbib}
\tcbuselibrary{minted}
\usemintedstyle{xcode}
\setminted{
    fontfamily=helvetica
}

\usepackage{booktabs}

\usepackage{outlines}
\usepackage{hyperref}
\usepackage{graphicx}
\usepackage{amsmath}
\usepackage{makecell}

% justify
\usepackage{ragged2e}
\justifying\let\raggedright\justifying


% outline space
\makeatletter
\renewenvironment{outline}[1][]{%
  \ifthenelse{\equal{#1}{}}{}{\renewcommand{\ol@type}{#1}}%
  \ol@z%
  \newcommand{\0}{\ol@toz\ol@z}%
  \newcommand{\1}{\vspace{\dimexpr\outlinespacingscalar\baselineskip-\baselineskip}\ol@toi\ol@i\item}%
  \newcommand{\2}{\ol@toii\ol@ii\item}%
  \newcommand{\3}{\ol@toiii\ol@iii\item}%
  \newcommand{\4}{\ol@toiiii\ol@iiii\item}%
}{%
  \ol@toz\ol@exit%
}
\makeatother




\renewcommand\theadalign{bc}
\renewcommand\theadfont{\bfseries}
\renewcommand\theadgape{\Gape[4pt]}
\renewcommand\cellgape{\Gape[4pt]}

\beamertemplatenavigationsymbolsempty

\newif\ifsectiontitlepage
\sectiontitlepagetrue

\newcommand{\nosectiontitlepage}{\sectiontitlepagefalse}


\AtBeginSection[]{
\ifsectiontitlepage
  \begin{frame}
  \tableofcontents[currentsection, hideothersubsections]
  \end{frame}
\fi
}



%%

% color
\definecolor{coreFontMain}{RGB}{26,26,26}
\definecolor{coreFontSub}{RGB}{106,106,106}
\definecolor{coreBackgroundLight}{RGB}{230,230,230}
\definecolor{coreBackground}{RGB}{255,255,255}

% heading
\newlength{\headerheight}
\setlength{\headerheight}{0.094\paperheight}


\setbeamerfont{frametitle}{size=\fontsize{12}{14.4}}
\setbeamertemplate{frametitle}
{%
\nointerlineskip
\begin{beamercolorbox}[wd=\paperwidth,left,sep=0pt]{}
\begin{tcolorbox}[enhanced,frame hidden, 
    borderline south={0.5pt}{0pt}{coreFontSub},
    colback=coreBackground, coltext=coreFontSub,height=\headerheight,notitle,left*=0.05\paperwidth,boxsep=0pt,bottom=0pt,halign=left,valign=bottom,before skip=0pt,after skip=0pt, before upper=\strut]
    \usebeamercolor{frametitle}\insertframetitle
\end{tcolorbox}
\end{beamercolorbox}
}


% footer
\newlength{\footerheight}
\setlength{\footerheight}{0.06\paperheight}
\setbeamercolor{footer}{bg=coreBackgroundLight, fg=coreFontMain}

\setbeamerfont{footline}{size=\fontsize{10}{12}}
\setbeamertemplate{footline}
{%
\begin{beamercolorbox}[sep=0pt,wd=\paperwidth,center]{footer}
%\begin{minipage}[c][\footerheight]{\textwidth}
\begin{columns}
    \begin{column}{0.7\textwidth}
        \begin{minipage}[c][\footerheight]{\textwidth}
            {\insertsectionhead}             
        \end{minipage}
    \end{column}
    \begin{column}{0.1\textwidth}
        \begin{minipage}[c][\footerheight]{\textwidth}
            \centering
            \insertframenumber/\inserttotalframenumber            
        \end{minipage}
    \end{column}
\end{columns}
    
%\end{minipage}
\end{beamercolorbox}
}



% Title Page

% header
% \newlength{\titleheaderheight}
% \setlength{\titleheaderheight}{0.0666667\paperheight}
% \setbeamercolor{header}{bg=coreBackgroundLight}

%% logo
\newlength{\logoheight}
\setlength{\logoheight}{0.297\paperheight}

%% title
\setbeamercolor{title}{fg=coreFontMain}
\setbeamerfont{title}{size=\fontsize{18.14}{21.768}}
\newlength{\titleheight}
\setlength{\titleheight}{0.19\paperheight}


%% meta
\newlength{\metaheight}
\setlength{\metaheight}{0.21\paperheight}
\setbeamercolor{meta}{fg=coreFontSub}
\setbeamerfont{meta}{size=\fontsize{9.07}{10.884}}

%% contact
\newlength{\contactheight}
\setlength{\contactheight}{0.193333\paperheight}



\makeatletter
% \newcommand{\iftitlegraphicisempty}{%
%   \ifx\@empty\inserttitlegraphic%
%     \expandafter\@firstoftwo%
%   \else%
%     \expandafter\@secondoftwo%
%   \fi%
% }
\setbeamertemplate{title page}{
% \nointerlineskip
% 
\ifx\@empty\inserttitlegraphic%
    \setlength{\titleheight}{0.4\paperheight}
    \setlength{\metaheight}{0.4\paperheight}%
    \setbeamerfont{meta}{size=\fontsize{11}{13.2}}
\else%
    \inserttitlegraphic%
\fi%

\begin{tcolorbox}[enhanced,frame hidden, borderline south={0.5pt}{0pt}{coreFontMain}, colback=coreBackground,height=\titleheight,boxsep=0pt,notitle,halign=flush center,valign=center, bottom=-30pt,before skip=0pt,after skip=0pt, coltext=coreFontMain]
\usebeamercolor{title}
\usebeamerfont{title}\inserttitle
\end{tcolorbox}

\usebeamercolor[fg]{meta}{
\begin{minipage}[c][\metaheight]{\textwidth}
    \usebeamerfont{meta}
    \centering
    \insertauthor \linebreak
    \insertinstitute \linebreak
    \insertdate
\end{minipage}
}


\begin{tcolorbox}[enhanced,frame hidden, colback=coreBackground,height=\contactheight,notitle,halign=flush center,valign=top,before skip=0pt,after skip=0pt,top=-5pt]
\begin{tabular}{ccc}
\href{mailto:zhangbohan@buaa.edu.cn}{\includegraphics[height=3mm]{bohanbeamerstyle/figures/mail.png}} & \href{https://github.com/angelpone}{\includegraphics[height=3mm]{bohanbeamerstyle/figures/github.png}} & \href{https://twitter.com/zhangbhangel}{\includegraphics[height=3mm]{bohanbeamerstyle/figures/Twitter.png}}
\end{tabular}

\end{tcolorbox}


% \begin{beamercolorbox}[sep=2pt,center,wd=\paperwidth,ht=\footerheight]{footer}
% \tiny
% \insertframenumber/\inserttotalframenumber
% \end{beamercolorbox}
% \nointerlineskip
}


% 
\setbeamercolor{normal text}{bg=coreBackground,fg=coreFontMain}

% itemize
\setbeamertemplate{itemize items}[circle]

\setlength{\leftmarginii}{10pt}
\setbeamercolor{itemize item}{fg=coreFontMain}
\setbeamercolor{itemize subitem}{fg=coreFontMain}
\setbeamercolor{itemize subsubitem}{fg=coreFontMain}




% \titlegraphic{ss}
\title{Discrete Forecast Reconciliation}
\author{Bohan Zhang}
\institute{Beihang University \\ 
KLLAB seminar}
\date{April 20, 2023}


% \nosectiontitlepage

\begin{document}

\begin{frame}[plain]

    \maketitle

\end{frame}



\begin{frame}
    \frametitle{Collaborate with}
    \fontsize{9}{10.8}\selectfont
    \begin{table}
    \begin{tabular}{ccc}
        \includegraphics[width=0.3\textwidth]{figures/tas.jpg} &
        \includegraphics[width=0.3\textwidth]{figures/yfkang.png} &
        \includegraphics[width=0.3\textwidth]{figures/fengli.png} \\
        \begin{minipage}[t]{0.3\textwidth}\centering Anastasios Panagiotelis \\ University of Sydney \end{minipage} &
        \begin{minipage}[t]{0.3\textwidth}\centering Yanfei Kang \\ Beihang University \end{minipage} &
        \begin{minipage}[t]{0.3\textwidth}\centering Feng Li \\ Central University of \\ Finance and Economics \end{minipage}
    \end{tabular}
    \end{table}

\end{frame}


\begin{frame}
    \frametitle{CONTENTS}
    \tableofcontents
\end{frame}


\section{Introduction}
\begin{frame}
\frametitle{Motivation: the problem}

\begin{outline}
\1 Non-negative time series with discrete values, particularly those with low counts, commonly arise in various fields. Examples include:
\2 occurrences of “black swan” events
\2 intermittent demand in the retail industry

\1 Hierarchical time series (HTS) is of great concern in these applications.

\1 {\color{red}However, research attention paied on count HTS forecasting is limited.}

\end{outline}

\end{frame}

\begin{frame}
\frametitle{Motivation: the challenge}

\begin{outline}
\1 The \textbf{forecast reconciliation framework} can produce coherent forecasts and has been proven to enhance forecast accuracy by utilizing forecast combination.
\1 Forecast reconciliation was designed for continuous-valued HTS based on projection and \textbf{can not be directly applied to} discrete-valued HTS.
\2 Forecasts should be “coherent” such that their support matches the support of the variable (\Citealp{freelandForecastingDiscreteValued2004})
\2 Transformation from continuous forecasts to integer decision introduces additional operational costs.

\end{outline}

\end{frame}


\begin{frame}
\frametitle{Motivation: the idea}

To address these concerns, we propose a discrete forecast reconciliation framework which

\begin{itemize}
    \item first produces probabilistic forecasts for each series, then obtains coherent \textbf{joint predictive distribution} for the HTS through reconciliation,
    \item allows for the employment of forecasting methods for univariate count time series in the literature,
    \item utilises scoring rules such as Brier Score to evaluate the forecasts and train the reconciliation matrix.
\end{itemize}

\end{frame}

\section{Related work}
\begin{frame}
\frametitle{Related Work}

A series of work on forecast reconciliation for count HTSs:

\begin{itemize}
    \fontsize{9}{10.2}\selectfont
    \item Corani, G., Azzimonti, D., Rubattu, N., \& Antonucci, A. (2022). Probabilistic Reconciliation of Count Time Series (arXiv:2207.09322). arXiv.
    \item Zambon, L., Azzimonti, D., \& Corani, G. (2022). Efficient probabilistic reconciliation of forecasts for real-valued and count time series (arXiv:2210.02286). arXiv.
    \item Zambon, L., Agosto, A., Giudici, P., \& Corani, G. (2023). Properties of the reconciled distributions for Gaussian and count forecasts (arXiv:2303.15135). arXiv.
\end{itemize}

The proposed framework conditions base probabilistic forecasts of the most disaggregated series on base forecasts of aggregated series. However, it fails to restore the dependence structure within hierarchical time series.


\end{frame}


\section{The discrete forecast reconciliation framework}

\begin{frame}
\frametitle{Coherent and incoherent domain for discrete HTS}
\begin{table}
    \fontsize{9}{10.2}\selectfont
\begin{tabular}{ll}
    \toprule
    HTS & $\mathbf{Y}=(Y_1,Y_2,\dots,Y_n)'$ \\
    basis time series & $(Y_1,Y_2,\dots,Y_{m})'$ \\
    domain of $i$-th variable & $\mathcal{D}(Y_i) = \{0,1,\dots,D_i\}$ \\
    Incoherent domain of HTS & $\hat{\mathcal{D}}(\mathbf{Y}) = \{0,\dots,D_1\}\times \dots\times \{0,\dots,D_n\}$ \\
    Coherent domain of HTS & $\tilde{\mathcal{D}}(\mathbf{Y}) = \{0,\dots,D_1\}\times \dots\times\{0,\dots,D_m\}$ \\\bottomrule
\end{tabular}
\end{table}

\begin{itemize}
    \item Incoherent domain is the Cartesian product of domains of all variables, whose elements are possibly incoherent.
    \item Coherent domain is a subset of incoherent domain whose elements are \textit{coherent}.
\end{itemize}

\end{frame}


\begin{frame}
\frametitle{Example}
\begin{block}{Variables}
\[
\begin{aligned}
  \mathcal{D}(Y_1) = \{0, 1\}, \mathcal{D}(Y_2) = \{0, 1\}, \\
  Y_3=Y_1+Y_2, \mathcal{D}(Y_3)\in\{0,1,2\}
\end{aligned}
\]
\end{block}


\begin{block}{Incoherent domain}
\[
\begin{aligned}
\hat{\mathcal D}(\mathbf{Y})=&\left\{\mathbf{(0,0,0)'},(0,1,0)',(1,0,0)',(1,1,0)',\right.\\
&\left.(0,0,1)',\mathbf{(0,1,1)'},\mathbf{(1,0,1)'},(1,1,1)',\right.\\
&\left.(0,0,2)',(0,1,2)',(1,0,2)',\mathbf{(1,1,2)'}\right\}\,,
\end{aligned}
\]
\end{block}


\begin{block}{Coherent domain}
\[
    \tilde{\mathcal D}(\mathbf{Y})=\left\{(0,0,0)',(0,1,1)',(1,0,1)',(1,1,2)'\right\}\,.
\]    
    
\end{block}
\end{frame}

\begin{frame}
\framtitle{Incoherent and coherent forecasts}
\end{frame}

\section{Score-optimal algorithm: DFR}

\section{Addressing the dimensionality problem: SDFR}

\section{Simulation}

\section{Emprical study}

\section{Conclusion}



\bibliography{../manuscript/references.bib}
\bibliographystyle{agsm}
\end{document}